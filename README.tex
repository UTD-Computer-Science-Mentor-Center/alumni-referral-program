\documentclass[letterpaper, 12pt]{article}
% \setlength{\parindent}{30pt}
\usepackage{ifluatex}
\usepackage{ifpdf}
\usepackage[T1]{fontenc}
\usepackage[margin=1in]{geometry} % Page Margins
% \usepackage{bookmark}
\usepackage{svg}

\usepackage{parskip}

\ifluatex{}
    \usepackage[luatex]{hyperref}

\else\ifxetex{}
    \usepackage[xetex]{hyperref}
\else
    \usepackage[pdftex]{hyperref}

    % Enable correct encoding
    \usepackage[utf8]{inputenc}
\fi\fi
\hypersetup{hidelinks}

\overfullrule=2cm %To check for overfull \hbox


% Configure Ordered Lists
\usepackage{enumitem}
\setlist[enumerate]{nosep,
    label=\arabic*\char41,
}

% \usepackage{titlesec}
% \titleformat{\section}
%     % Section title format
%     {\bfseries\large}
%     % Prefix text
%     {}
%     % Space between prefix and section title
%     {0pt}
%     % Section title text
%     {}
%     % After section formatting
%     [\vspace{-2ex}]

\pagestyle{empty} % Remove Page Numbers

\newcommand{\pratikMinSemesters}{2}
\newcommand{\bossMinSemesters}{3}
\newcommand{\currentBoss}{Jason Smith}
\newcommand{\QA}[2]{\textit{#1}\medskip

    #2\bigskip
}
\newcommand{\note}[1]{\textbf{Note:}~#1}

\newlist{simpleList}{description}{2}
\setlist[simpleList]{leftmargin=!}
\newcommand{\requesterCondition}[2]{\begin{simpleList}[labelwidth=19ex]
        \item[#1]#2
    \end{simpleList}
}

\begin{document}

\begin{center}
    \bfseries\Large CSMC Referral Program
\end{center}

\section{Disclaimer}\label{section:disclaimer}

This program does not guarantee a referral, interview, nor offer. At the
bare minimum, you are guaranteed to have at least one person looking at your
resume.

If at any point you provide your contact information outside the confines of
this program, you are responsible for any messages and/or spam you
receive.\label{section:disclaimer:spamMessages}

\section{Problem Statement}

It was commonplace for fellow CSMC mentors to give referrals to other mentors.
As mentors come and go, this positive feedback loop stays.

The global pandemic changed everything. The cycle halted. Many lost internship
opportunities. May the odds be ever in your favor to find a full time job out of
college.

\section{Mission Statement}

This program is one answer that seeks to provide a semi-established channel for
(current and even former) CSMC mentors find internships and their first job after
graduating.

While it may seem like a temporarily solution to a bigger problem, it is still
walking towards a more permanent, establish path.

\section{Participation Requirements}

\subsection{Requester}

You are eligible to request referrals if and only if you meet at least one of
the following classifications:

\begin{enumerate}
    \item\requesterCondition{Current Student}{You are a student at UTD\@. You
        have started at least semester \bossMinSemesters{} at the CSMC.
    }
    \item\requesterCondition{Senior Student}{You are a student at UTD\@. You are
        in/starting your last year before graduating. You have worked at least
        \pratikMinSemesters{} complete semesters at the CSMC.
    }
    \item\requesterCondition{Alum}{You have graduated from UTD\@. You have
        worked at least \pratikMinSemesters{} complete semesters at the CSMC.
    }
\end{enumerate}

\subsection{Referrer}

Requirements are still a work in progress.

For now, it is only former CSMC mentors; however, this is what you would more
likely call ``guidelines'' than actual rules.

\section{Opt-in Process}

\subsection{Requester}

At some point \currentBoss{} should have posted Pratik Bhusal's referral request
form. Assuming you are eligible, please fill that form out to show your interest
in the program.

\subsection{Referrer}

If you want to give referrals and Pratik Bhusal has not contacted you already,
notify Pratik Bhusal through the github issue request page or through Pratik
Bhusal's email address associated with this program.

\note{%
    As mentioned in the \hyperref[section:disclaimer:spamMessages]{disclaimer},
    if you provide your contact information to a requester, you are responsible
    for any communication that does not go through this program.
}

\section{Opt-out Process}

If you no longer wish to participate, please let Pratik Bhusal know.

\section{Potential/Frequently Asked Questions}

\begin{enumerate}[leftmargin=*]
    \item\QA%
        {When is the best time to apply/participate?}
        {As early as possible.}
    \item\QA%
        {%
            Why can't you just post all the referrers' contact information in
            the primary method of communication used by the CSMC (e.g: Slack)?
        }
        {%
            The referrer has only guaranteed sharing contact information with
            Pratik Bhusal. Them receiving unsolicited spam from a random
            individual should never happen.
        }
    \item\QA%
        {%
            If I am a student at UTD and just finished my second semester
            working at the CSMC, am I eligible to participate?
        }
        {%
            If you are in your last semester/year before graduating UTD, you are
            classified as a senior student and hence eligible.

            If you are \textbf{not} in your last semester/year before graduating
            UTD, you are \textbf{not} eligible.
        }
    \item\QA%
        {%
            What benefit(s) do referrers get?
        }
        {%
            They provide referrals for any combination of reasons. It could be
            because they want to help their former coworkers, there is some
            financial incentive, or both.

            In the case of a financial incentive, companies \textit{may} provide
            a financial incentive to employees for successful candidate hirings.
        }
    \item\QA%
        {How is my information stored?}
        {
            It is stored on Google spreadsheets. Automatic data encryption is
            desired, but still in the brainstorming phase.
        }
\end{enumerate}


\section{Referral Request Process Diagram}

\includesvg[inkscapelatex=false,width=\linewidth]{./graphics/process}\label{fig:requestProcess}

\note{%
    Special conditions may exist based on the referrer's preferences and/or
    restrictions. These conditions will be mentioned in the provided list.
}

\note{Referral requests will come at a first come first serve basis.}

\note{%
    In the situation that a company on that list has multiple potential
    referrers, one will be randomly picked.
}

\end{document}
